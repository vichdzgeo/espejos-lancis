%% Generated by Sphinx.
\def\sphinxdocclass{report}
\documentclass[letterpaper,10pt,spanish]{sphinxmanual}
\ifdefined\pdfpxdimen
   \let\sphinxpxdimen\pdfpxdimen\else\newdimen\sphinxpxdimen
\fi \sphinxpxdimen=.75bp\relax

\PassOptionsToPackage{warn}{textcomp}
\usepackage[utf8]{inputenc}
\ifdefined\DeclareUnicodeCharacter
% support both utf8 and utf8x syntaxes
\edef\sphinxdqmaybe{\ifdefined\DeclareUnicodeCharacterAsOptional\string"\fi}
  \DeclareUnicodeCharacter{\sphinxdqmaybe00A0}{\nobreakspace}
  \DeclareUnicodeCharacter{\sphinxdqmaybe2500}{\sphinxunichar{2500}}
  \DeclareUnicodeCharacter{\sphinxdqmaybe2502}{\sphinxunichar{2502}}
  \DeclareUnicodeCharacter{\sphinxdqmaybe2514}{\sphinxunichar{2514}}
  \DeclareUnicodeCharacter{\sphinxdqmaybe251C}{\sphinxunichar{251C}}
  \DeclareUnicodeCharacter{\sphinxdqmaybe2572}{\textbackslash}
\fi
\usepackage{cmap}
\usepackage[T1]{fontenc}
\usepackage{amsmath,amssymb,amstext}
\usepackage{babel}
\usepackage{times}
\usepackage[Sonny]{fncychap}
\ChNameVar{\Large\normalfont\sffamily}
\ChTitleVar{\Large\normalfont\sffamily}
\usepackage{sphinx}

\fvset{fontsize=\small}
\usepackage{geometry}

% Include hyperref last.
\usepackage{hyperref}
% Fix anchor placement for figures with captions.
\usepackage{hypcap}% it must be loaded after hyperref.
% Set up styles of URL: it should be placed after hyperref.
\urlstyle{same}
\addto\captionsspanish{\renewcommand{\contentsname}{Yúcatan:}}

\addto\captionsspanish{\renewcommand{\figurename}{Figura }}
\makeatletter
\def\fnum@figure{\figurename\thefigure{}}
\makeatother
\addto\captionsspanish{\renewcommand{\tablename}{Tabla }}
\makeatletter
\def\fnum@table{\tablename\thetable{}}
\makeatother
\addto\captionsspanish{\renewcommand{\literalblockname}{Lista}}

\addto\captionsspanish{\renewcommand{\literalblockcontinuedname}{proviene de la página anterior}}
\addto\captionsspanish{\renewcommand{\literalblockcontinuesname}{continué en la próxima página}}
\addto\captionsspanish{\renewcommand{\sphinxnonalphabeticalgroupname}{Non-alphabetical}}
\addto\captionsspanish{\renewcommand{\sphinxsymbolsname}{Símbolos}}
\addto\captionsspanish{\renewcommand{\sphinxnumbersname}{Numbers}}

\addto\extrasspanish{\def\pageautorefname{página}}

\setcounter{tocdepth}{0}



\title{Modelo de vulnerabilidad y capacidad adaptativa}
\date{19 de marzo de 2021}
\release{}
\author{LANCIS}
\newcommand{\sphinxlogo}{\vbox{}}
\renewcommand{\releasename}{}
\makeindex
\begin{document}

\ifdefined\shorthandoff
  \ifnum\catcode`\=\string=\active\shorthandoff{=}\fi
  \ifnum\catcode`\"=\active\shorthandoff{"}\fi
\fi

\pagestyle{empty}
\sphinxmaketitle
\pagestyle{plain}
\sphinxtableofcontents
\pagestyle{normal}
\phantomsection\label{\detokenize{index::doc}}



\chapter{Exposición}
\label{\detokenize{exposicion:exposicion}}\label{\detokenize{exposicion::doc}}

\section{Modelo multicriterio}
\label{\detokenize{exposicion:modelo-multicriterio}}
\noindent{\hspace*{\fill}\sphinxincludegraphics{{vulnerabilidad_exposicion_13abr2020}.png}\hspace*{\fill}}


\section{Insumos}
\label{\detokenize{exposicion:insumos}}

\subsection{Biológica}
\label{\detokenize{exposicion:biologica}}

\subsubsection{Vegetación acuatíca}
\label{\detokenize{exposicion:vegetacion-acuatica}}\begin{quote}

\sphinxstyleemphasis{Exposición - biológica}

\sphinxstylestrong{Peso local}:0.16

\sphinxstylestrong{insumo:} 

\sphinxstylestrong{Definición:} Distancia entre los pastos y la línea de costa

\sphinxcode{\sphinxupquote{mínimo: 0.0
máximo:3000}}

\sphinxstylestrong{tipo de función:}  Continua - Logística

\begin{sphinxadmonition}{note}{Nota:}
Esta capa fue procesada en grass 7 ya que no se pudo procesar en la
plataforma
\end{sphinxadmonition}

archivo json: fv\_exp\_bio\_v\_acuatica.json

\sphinxcode{\sphinxupquote{centro:1500,
min:0,
max:3000,
saturacion: 3,
k:0.0834999999999999}}

\noindent\sphinxincludegraphics{{fv_c_exp_veg_acuatica}.png}

\sphinxstylestrong{Resultado:} SIG/desarrollo/sig\_papiit/entregables/exposicion/biologica/v\_acuatica\_yuc/fv\_v\_acuatica\_yuc.tif

\sphinxstylestrong{Nombre de la capa:} 

\sphinxstylestrong{issue:} 
\end{quote}


\subsubsection{Vegetación costera}
\label{\detokenize{exposicion:vegetacion-costera}}
\sphinxstyleemphasis{Exposición - Biológica}

Criterio : Vegetación costera

\sphinxstylestrong{Definición}: Tipo de vegetación que se encuentra en la franja de 3 kilometros (manglar, dunas)

\sphinxstylestrong{peso local:} 0.84

Esta capa representa la integración de las capas de distancia de dunas costeras con la capa de distancias de manglar
la construcción es

\sphinxcode{\sphinxupquote{fv\_v\_costera\_yuc = fv\_distancia\_dunas.tif*0.25 + fv\_ditancia\_manglar.tif*0.75}}

\sphinxstylestrong{Insumos}
\begin{quote}

\sphinxstylestrong{Dunas costeras}

\sphinxstyleemphasis{Exposición - biológica - Vegetación costera}
\begin{quote}

\sphinxstylestrong{peso}: 0.25

\sphinxstylestrong{insumo:} ifv\_distancia\_dunas\_yuc.tif

\sphinxstylestrong{Definición:} Distancia mínima a una celda de dunas

\sphinxcode{\sphinxupquote{mínimo: 0.0  máximo:3000}}

\sphinxstylestrong{Tipo de función:} Continua - logística

archivo json: fv\_exp\_bio\_veg\_dunas.json

\sphinxcode{\sphinxupquote{centro:100
min:0
max:3000
saturación:10
k:0.255}}

\noindent\sphinxincludegraphics{{fv_c_exp_dunas_costeras}.png}

\sphinxstylestrong{Resultado:}  SIG/desarrollo/sig\_papiit/entregables/exposición/biologica/v\_costera\_yuc/fv\_distancia\_dunas\_yuc.tif
\end{quote}

\sphinxstylestrong{Manglar}

\sphinxstyleemphasis{Exposición - Biológica - Vegetación costera}
\begin{quote}

\sphinxstylestrong{peso local:} 0.75

insumo: ifv\_distancia\_manglar\_yuc.tif

\sphinxstylestrong{Definición:} Distancia mínima a una celda de manglar, independiente de la orientación. Barrera a inundaciones

\sphinxstylestrong{Tipo de función:} Continua - logística

archivo json: fv\_exp\_bio\_veg\_manglar.json

\sphinxcode{\sphinxupquote{centro:250
min:0
max:3000
saturación:4
k:0.108}}

\noindent\sphinxincludegraphics{{fv_c_exp_manglar}.png}

\sphinxstylestrong{Resultado:} /SIG/desarrollo/sig\_papiit/entregables/exposición/biologica/v\_costera\_yuc/fv\_distancia\_manglar\_yuc.tif
\end{quote}
\end{quote}

\sphinxstylestrong{Resultado}: SIG/desarrollo/sig\_papiit/entregables/exposición/biologica/v\_costera\_yuc/fv\_v\_costera\_distancia\_yuc.tif

\sphinxstylestrong{Nombre de la capa:} fv\_v\_costera\_distancia\_yuc

\sphinxstylestrong{issue}: 


\subsubsection{Resultado}
\label{\detokenize{exposicion:resultado}}
Los insumos de exposición biológica se integran en una capa de la siguiente forma:

\sphinxcode{\sphinxupquote{exp\_biologica = fv\_costera\_distacia\_yuc * 0.84 + fv\_v\_acuatica\_yuc * 0.16}}

Ruta : SIG/desarrollo/sig\_papiit/entregables/exposición/salida/exp\_biologica.tif


\subsection{Física}
\label{\detokenize{exposicion:fisica}}

\subsubsection{Distancia a la playa}
\label{\detokenize{exposicion:distancia-a-la-playa}}
\sphinxstyleemphasis{Exposición - Física}
\begin{quote}

\sphinxstylestrong{peso local:} 0.13

\sphinxstylestrong{insumo:} 

\sphinxstylestrong{Definición:} Distancia a la playa

\sphinxcode{\sphinxupquote{mínimo: 0.00
máximo:3000.00}}

\sphinxstylestrong{Tipo de función:} Continua - logística

archivo json: fv\_exp\_fis\_playa.json

\sphinxcode{\sphinxupquote{centro:60,
min:0,
max:3000,
saturación:7,
k:0.1815}}

\noindent\sphinxincludegraphics{{fv_c_exp_aplaya}.png}
\end{quote}

\sphinxstylestrong{Resultado}: SIG/desarrollo/sig\_papiit/entregables/exposición/biologica/v\_costera\_yuc/fv\_distancia\_playa\_yuc.tif

\sphinxstylestrong{Nombre de la capa:} 

\sphinxstylestrong{issue}: 


\subsubsection{Elevación}
\label{\detokenize{exposicion:elevacion}}
\sphinxstyleemphasis{Exposición - Física}
\begin{quote}

\sphinxstylestrong{peso local:} 0.87

\sphinxstylestrong{insumo:} 

\sphinxstylestrong{Definición:} Localización sobre el nivel medio del mar

\sphinxstylestrong{Tipo de función:} Continua - Concava decreciente

arhivo json: fv\_exp\_fis\_elevacion.json

\sphinxcode{\sphinxupquote{min: 0
max: 31
gama: 0.049249999999999995
saturacion: 3}}

\noindent\sphinxincludegraphics{{fv_c_exp_elevacion}.png}
\end{quote}

\sphinxstylestrong{Resultado}: SIG/desarrollo/sig\_papiit/entregables/exposición/fisica/elev\_yuc/fv\_elevacion\_yuc.tif

\sphinxstylestrong{Nombre de la capa:} 

\sphinxstylestrong{issue}: 


\subsubsection{Resultado:}
\label{\detokenize{exposicion:id1}}
Los insumos de exposición - física se integran en una capa de la siguiente forma:

\sphinxcode{\sphinxupquote{exp\_fisica = fv\_distancia\_playa * 0.13 + fv\_elevacion * 0.87}}

\sphinxstylestrong{Ruta:} SIG/desarrollo/sig\_papiit/entregables/exposición/salida/exp\_fisica.tif


\section{Integración}
\label{\detokenize{exposicion:integracion}}
el criterio de \sphinxstyleemphasis{Biológica} y \sphinxstyleemphasis{Física} se integran  para formar la capa de exposición

exposición = exp\_biologica * 0.50 + exp\_fisica * 0.50

\sphinxstylestrong{Ruta:} SIG/desarrollo/sig\_papiit/entregables/exposición/salida/exposicion\_yuc.tif


\chapter{Susceptibilidad}
\label{\detokenize{sensibilidad:susceptibilidad}}\label{\detokenize{sensibilidad::doc}}

\section{Modelo multicriterio}
\label{\detokenize{sensibilidad:modelo-multicriterio}}
\noindent{\hspace*{\fill}\sphinxincludegraphics{{vulnerabilidad_susceptibilidad_1sep2020_fv}.png}\hspace*{\fill}}


\section{Insumos}
\label{\detokenize{sensibilidad:insumos}}

\subsection{Biológica}
\label{\detokenize{sensibilidad:biologica}}

\subsubsection{Vegetación costera}
\label{\detokenize{sensibilidad:vegetacion-costera}}\begin{quote}

\sphinxstyleemphasis{susceptibilidad - Biológica}

\sphinxstylestrong{peso local} : 0.66

\sphinxstylestrong{Definición:} integración de las capas de presencia de vegetación de duna constera con la capa de manglar.
\end{quote}


\paragraph{Insumos}
\label{\detokenize{sensibilidad:id1}}\begin{quote}

\sphinxstylestrong{Manglar}
\begin{quote}

\sphinxstyleemphasis{susceptibilidad - Biológica - Vegetación costera}

\sphinxstylestrong{peso local:} 1.0

\sphinxstylestrong{Insumo:}  ifv\_manglar\_presencia\_yuc.tif

\sphinxstylestrong{Definición:} presencia de manglar en la zona de estudio

\sphinxstylestrong{Tipo de función:} Binaria
\end{quote}

\sphinxstylestrong{Duna costera}

\sphinxstyleemphasis{susceptibilidad - Biológica - Vegetación costera}
\begin{quote}

\sphinxstylestrong{peso local:}: 0.25

\sphinxstylestrong{Insumo}: ifv\_dunas\_presencia\_yuc.tif

\sphinxstylestrong{Definición:} presencia de vegetacion de dunas costeras en la zona de estudio

\sphinxstylestrong{Tipo de función:} Binaria
\end{quote}
\end{quote}

\sphinxstylestrong{Resultado:}: SIG/desarrollo/sig\_papiit/entregables/susceptibilidad/biologica/v\_costera\_yuc/fv\_v\_costera\_presencia\_yuc\_100m.tif

\sphinxstylestrong{Nombre de la capa:} 

\sphinxstylestrong{issue}:  


\subsection{Física}
\label{\detokenize{sensibilidad:fisica}}

\subsubsection{Dunas costeras}
\label{\detokenize{sensibilidad:dunas-costeras}}
\sphinxstyleemphasis{susceptibilidad - Física}
\begin{quote}

\sphinxstylestrong{Peso local}:0.56

\sphinxstylestrong{Insumo:} 

\sphinxstylestrong{Definición:} Esta capa representa la presencia de dunas costeras en la costa, El insumo ocupado
para la generación de esta capa corresponde a información del POETY combinado con
el ancho de playa.

\sphinxstylestrong{Tipo de función:} Binaria

\sphinxstylestrong{Resultado:} C:/Dropbox (LANCIS)/SIG/desarrollo/sig\_papiit/entregables/susceptibilidad/fisica/duna\_yuc/fv\_duna\_yuc/fv\_duna\_yuc\_100m.tif

\sphinxstylestrong{Nombre de la capa:} 

\sphinxstylestrong{issue} 
\end{quote}


\subsubsection{Elevación}
\label{\detokenize{sensibilidad:elevacion}}
\sphinxstyleemphasis{susceptibilidad - Física}
\begin{quote}

\sphinxstylestrong{Peso local:}:0.04

\sphinxstylestrong{Insumo:} 

\sphinxstylestrong{Definición:} El insumo para la generación de esta capa proviene del Continuo de Elevaciones Mexicano (CEM) de INEGI,
a la cual se le aplico una funcion de valor \sphinxstyleemphasis{concava creciente}

\sphinxstylestrong{Tipo de función:} continua - Concava creciente

arhivo json:

\sphinxcode{\sphinxupquote{min:0
max: 31
gama: 0.01975
saturacion: 13}}

\noindent\sphinxincludegraphics{{fv_c_sens_elevacion}.png}

\sphinxstylestrong{Resultado:} SIG/desarrollo/sig\_papiit/entregables/susceptibilidad/fisica/elev\_yuc/fv\_elevacion\_yuc\_100m.tif

\sphinxstylestrong{Nombre de la capa:} 

\sphinxstylestrong{issue} 
\end{quote}


\subsubsection{Tipo de litoral}
\label{\detokenize{sensibilidad:tipo-de-litoral}}
\sphinxstyleemphasis{susceptibilidad - Física}
\begin{quote}

\sphinxstylestrong{Peso local:} 0.07

\sphinxstylestrong{Insumo}: ifv\_tipo\_litoral\_yuc.tif

\sphinxstylestrong{Definición:} Esta capa representa la presencia de diferentes tipos de litoral, estos fueron clasificados
conforme a la siguiente tabla, el insumo ocupado es la capa de uso de suelo y vegetación
serie VI de INEGI

\sphinxstylestrong{Tipo de función:} Discreta


\begin{savenotes}\sphinxattablestart
\centering
\begin{tabulary}{\linewidth}[t]{|T|T|}
\hline
\sphinxstyletheadfamily 
Tipo de Vegetación
&\sphinxstyletheadfamily 
Tipo de litoral
\\
\hline
Área desprovista de vegetación
&
Arenoso
\\
\hline
Sin vegetación aparente
&
Arenoso
\\
\hline
vegetación dunas costeras
&
Arenoso
\\
\hline
Acuícola
&
Artificial
\\
\hline
Urbano construido
&
Artificial
\\
\hline
Agua
&
Lodoso
\\
\hline
Peten
&
Lodoso
\\
\hline
Vegetación halofila, hidrofila
&
Lodoso
\\
\hline
Agricultura de riego anual
&
Vegetado
\\
\hline
Agricultura de riego permanente
&
Vegetado
\\
\hline
Manglar
&
Vegetado
\\
\hline
Palmar, pastizal,manglar, tular
&
Vegetado
\\
\hline
Vegetación secundaria (5 clases)
&
Vegetado
\\
\hline
\end{tabulary}
\par
\sphinxattableend\end{savenotes}

Quendando de esta forma los pesos asignados para cada categoría.


\begin{savenotes}\sphinxattablestart
\centering
\begin{tabulary}{\linewidth}[t]{|T|T|T|}
\hline
\sphinxstyletheadfamily 
Categoria
&\sphinxstyletheadfamily 
Descripción
&\sphinxstyletheadfamily 
fv
\\
\hline
1
&
Arenoso
&
0.55
\\
\hline
2
&
Artificial
&
0.11
\\
\hline
3
&
Lodoso
&
1.00
\\
\hline
4
&
Vegetado
&
0.31
\\
\hline
\end{tabulary}
\par
\sphinxattableend\end{savenotes}

\sphinxstylestrong{Resultado:} SIG/desarrollo/sig\_papiit/entregables/susceptibilidad/fisica/t\_litoral\_yuc/fv\_tipo\_litoral\_yuc\_100m\_corregida.tif

\sphinxstylestrong{Nombre de la capa:} 

\sphinxstylestrong{issue} 
\end{quote}


\chapter{Resiliencia}
\label{\detokenize{resiliencia:resiliencia}}\label{\detokenize{resiliencia::doc}}

\section{Modelo multicriterio}
\label{\detokenize{resiliencia:modelo-multicriterio}}
\noindent{\hspace*{\fill}\sphinxincludegraphics{{vulnerabilidad_resiliencia_fv_23nov2020}.png}\hspace*{\fill}}


\section{Insumos}
\label{\detokenize{resiliencia:insumos}}

\subsection{Biológica}
\label{\detokenize{resiliencia:biologica}}

\subsubsection{Biodiversidad}
\label{\detokenize{resiliencia:biodiversidad}}
\sphinxstyleemphasis{Resiliencia - Biológica}
\begin{quote}

Clasificación de tipos de vegetación costera (a mayor biodiversidad, mayor resiliencia)
ejemplo: manglar, pastos marinos, vegetación duna, otros tipos de vegetación sumergible

\sphinxstylestrong{peso local:} 0.50

\sphinxstylestrong{insumo:} 

\sphinxstylestrong{Definición:} Esta capa representa la presencia de vegetación costera en la zona de estudio.

\sphinxstylestrong{Tipo de función:} Discreta

\noindent\sphinxincludegraphics{{fv_d_res_biodiversidad}.png}


\begin{savenotes}\sphinxattablestart
\centering
\begin{tabulary}{\linewidth}[t]{|T|T|T|}
\hline
\sphinxstyletheadfamily 
Categoria
&\sphinxstyletheadfamily 
Descripción
&\sphinxstyletheadfamily 
fv
\\
\hline
1
&
Dunas Costeras
&
0.33
\\
\hline
2
&
Manglar
&
1.00
\\
\hline
3
&
Tular
&
0.09
\\
\hline
4
&
Vegetación de peten
&
0.31
\\
\hline
5
&
vegetación arbustiva
&
0.06
\\
\hline
\end{tabulary}
\par
\sphinxattableend\end{savenotes}

\sphinxstylestrong{Resultado:} SIG/desarrollo/sig\_papiit/entregables/resiliencia/biologica/biodiversidad\_yuc/fv\_biodiversidad\_yuc\_100m.tif

\sphinxstylestrong{Nombre de la capa:} 

\sphinxstylestrong{issue:}  
\end{quote}


\subsubsection{Servicios ambientales}
\label{\detokenize{resiliencia:servicios-ambientales}}
Presencia de tipos de vegetación que proveen protección a la línea de costa y hábitat (pastos marinos, dunas costeras y manglar)

\sphinxstylestrong{Protección costera}

Presencia de tipos de vegetación que proveen protección a la línea de costa (dunas costeras, manglares, humedales)

\sphinxstyleemphasis{Resiliencia - Biológica - Servicios ambientales}

\sphinxstylestrong{Peso local:} 0.75

\sphinxstylestrong{Insumos:}
\begin{quote}

\sphinxstylestrong{Vegetación acúatica}
\begin{quote}

\sphinxstyleemphasis{Resiliencia - Biológica - Servicios ambientales - Protección costera}

\sphinxstylestrong{Peso local:} 0.16

\sphinxstylestrong{Insumo:} ifv\_v\_acuatica.tif

\sphinxstylestrong{Definición:} Distancia entre los pastos y la línea de costa

\sphinxstylestrong{Tipo de función:} continua - Logística invertida

\begin{sphinxadmonition}{note}{Nota:}
Esta capa fue procesada en grass 7
\end{sphinxadmonition}

archivo json: fv\_exp\_bio\_v\_acuatica.json

\sphinxcode{\sphinxupquote{centro:1500,
min:0,
max:3000,
saturacion: 3,
k:0.0834999999999999}}

\noindent\sphinxincludegraphics{{fv_c_res_veg_acuatica}.png}

\sphinxstylestrong{Resultado:} sig\_papiit/entregables/resiliencia/biologica/serv\_ambientales\_yuc/prot\_costera\_yuc/fv\_v\_acuatica\_yuc\_100m.tif
\end{quote}

\sphinxstylestrong{Entidades protectoras}
\begin{quote}

\sphinxstylestrong{Peso local:} 0.84

\sphinxstylestrong{Definición:} Presencia de manglar y Dunas costeras que ofrece

\sphinxstylestrong{Resultado:} sig\_papiit/entregables/resiliencia/biologica/serv\_ambientales\_yuc/prot\_costera\_yuc/fv\_entidades\_protectoras.tif

\noindent\sphinxincludegraphics{{fv_d_res_ent_protector}.png}
\end{quote}
\end{quote}

\sphinxstylestrong{Provisión}
\begin{quote}

\sphinxstyleemphasis{Resiliencia - Biológica - Servicios ambientales}

Presencia de tipos de vegetación que proveen alimento, materias primas, recursos genéticos

\sphinxstylestrong{Peso local}:0.50

\sphinxstylestrong{Insumo:} 

\sphinxstylestrong{Definición:}

\sphinxstylestrong{Tipo de función:} Discreta
\begin{quote}


\begin{savenotes}\sphinxattablestart
\centering
\begin{tabulary}{\linewidth}[t]{|T|T|T|}
\hline
\sphinxstyletheadfamily 
Categoria
&\sphinxstyletheadfamily 
Descripción
&\sphinxstyletheadfamily 
fv
\\
\hline
1
&
Dunas Costeras
&
0.32
\\
\hline
2
&
Manglar
&
1.00
\\
\hline
3
&
Tular
&
0.08
\\
\hline
4
&
Vegetación de peten
&
0.29
\\
\hline
5
&
vegetación arbustiva
&
0.05
\\
\hline
\end{tabulary}
\par
\sphinxattableend\end{savenotes}
\end{quote}

\sphinxstylestrong{Resultado:} /sig\_papiit/entregables/resiliencia/biologica/serv\_ambientales\_yuc/provision\_yuc/fv\_provision\_yuc\_100m.tif

\sphinxstylestrong{Nombre de la capa:}   \sphinxstylestrong{falta metadato}

\sphinxstylestrong{issue} 
\end{quote}


\subsection{Física}
\label{\detokenize{resiliencia:fisica}}

\subsubsection{Elevación}
\label{\detokenize{resiliencia:elevacion}}
\sphinxstyleemphasis{Resiliencia - Física}
\begin{quote}

\sphinxstylestrong{Peso local:}:0.60

\sphinxstylestrong{Insumo:} 

\sphinxstylestrong{Definición:}     El insumo para la generación de esta capa proviene del Continuo de Elevaciones Mexicano (CEM) de INEGI,
a la cual se le aplico una funcion de valor \sphinxstyleemphasis{concava decreciente}

\sphinxstylestrong{Tipo de función:} continua - Concava creciente
\begin{quote}

arhivo json:

\sphinxcode{\sphinxupquote{min: 0
max: 31
gama: 0.01975
saturacion:}}
\begin{quote}

\noindent\sphinxincludegraphics{{fv_c_res_elevacion}.png}
\end{quote}
\end{quote}

\sphinxstylestrong{Resultado:} SIG/desarrollo/sig\_papiit/entregables/resiliencia/fisica/elev\_yuc/fv\_elev\_yuc.tif

\sphinxstylestrong{Nombre de la capa:} 

\sphinxstylestrong{issue} 
\end{quote}


\subsubsection{Tipo de litoral}
\label{\detokenize{resiliencia:tipo-de-litoral}}
\sphinxstyleemphasis{Resiliencia - Física}
\begin{quote}

\sphinxstylestrong{Peso local}:0.40

\sphinxstylestrong{Insumo:} 

\sphinxstylestrong{Definición:} Esta capa representa la presencia de diferentes tipos de litoral, estos fueron clasificados
conforme a la siguiente tabla, el insumo ocupado es la capa de uso de suelo y vegetación
serie VI de INEGI

\sphinxstylestrong{Tipo de función:} Discreta
\begin{quote}


\begin{savenotes}\sphinxattablestart
\centering
\begin{tabulary}{\linewidth}[t]{|T|T|}
\hline
\sphinxstyletheadfamily 
Tipo de Vegetación
&\sphinxstyletheadfamily 
Tipo de litoral
\\
\hline
Área desprovista de vegetación
&
Arenoso
\\
\hline
Sin vegetación aparente
&
Arenoso
\\
\hline
vegetación dunas costeras
&
Arenoso
\\
\hline
Acuícola
&
Artificial
\\
\hline
Urbano construido
&
Artificial
\\
\hline
Agua
&
Lodoso
\\
\hline
Peten
&
Lodoso
\\
\hline
Vegetación halofila, hidrofila
&
Lodoso
\\
\hline
Agricultura de riego anual
&
Vegetado
\\
\hline
Agricultura de riego permanente
&
Vegetado
\\
\hline
Manglar
&
Vegetado
\\
\hline
Palmar, pastizal,manglar, tular
&
Vegetado
\\
\hline
Vegetación secundaria (5 clases)
&
Vegetado
\\
\hline
\end{tabulary}
\par
\sphinxattableend\end{savenotes}

Quendando de esta forma los pesos asignados para cada categoría.


\begin{savenotes}\sphinxattablestart
\centering
\begin{tabulary}{\linewidth}[t]{|T|T|T|}
\hline
\sphinxstyletheadfamily 
Categoria
&\sphinxstyletheadfamily 
Descripción
&\sphinxstyletheadfamily 
fv
\\
\hline
1
&
Arenoso
&
0.54
\\
\hline
2
&
Artificial
&
1.00
\\
\hline
3
&
Lodoso
&
0.19
\\
\hline
4
&
Vegetado
&
1.00
\\
\hline
\end{tabulary}
\par
\sphinxattableend\end{savenotes}
\end{quote}

\sphinxstylestrong{Resultado:} SIG/desarrollo/sig\_papiit/entregables/resiliencia/fisica/t\_litoral\_yuc/fv\_tipo\_litoral\_yuc\_v2.tif

\sphinxstylestrong{Nombre de la capa:} 

\sphinxstylestrong{issue} 
\end{quote}


\subsubsection{\sphinxstylestrong{Resultado}}
\label{\detokenize{resiliencia:resultado}}
\sphinxcode{\sphinxupquote{res\_fisica = fv\_ancho\_playa\_yuc * 0.62 + fv\_duna\_yuc * 0.27 + fv\_elev\_yuc * 0.06 + fv\_tipo\_litoral\_yuc * 0.05}}

Ruta: SIG/desarrollo/sig\_papiit/entregables/resiliencia/salida/res\_fisica.tif








\chapter{Vulnerabilidad}
\label{\detokenize{vulnerabilidad_yuc:vulnerabilidad}}\label{\detokenize{vulnerabilidad_yuc::doc}}
\noindent{\hspace*{\fill}\sphinxincludegraphics{{vulnerabilidad}.png}\hspace*{\fill}}



\renewcommand{\indexname}{Índice}
\printindex
\end{document}